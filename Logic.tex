\CapPkg provides two types of logic applied to objects and morphisms of a category.
In this chapter, the two layers are described together with some features of the logic.
We start by giving some general remarks, then describing the logic and the syntax explicitly.
After that, we give a description how to create your own logic files to the category.

\section{General remarks to logic in \CapPkg}

There are currently two types of logical propagation between \GAP objects implemented.
The first is a relation between different filters.

\begin{example}
 Every isomorphism is a monomorphism. So there is an implication between the properties
 \texttt{IsIsomorphism} and \texttt{IsMonomorphism}. Everytime the property \texttt{IsIsomorphism}
 is set to true, the property \texttt{IsMonomorphism} is also true.
\end{example}

In \CapPkg, such a propagation is handled by \texttt{TrueMethods}. That means those relations
are static and cannot been turn off, but also \GAP knows about them.

The second propagation is the propagation of predicates between input and output objects
of a certain method.

\begin{example}
  The composition of two monomorphisms is again a monomorphism. So, if for both input morphisms
  of the method \texttt{PreCompose} the property \texttt{IsMonomorphism} becomes set to \texttt{true},
  in the resulting morphism of \texttt{PreCompose}, \texttt{IsMonomorphism} is also set to \texttt{true}.
\end{example}

Internally, such propagations are carried out via \texttt{ToDoLists}. This means that the properties
do not need to be known when the mentioned operation is executed, but can be propagated later. As a drawback,
\GAP itself does not know about the relation before it is propagated, so it cannot take use of it.

\subsection{Switchability of Logic}

Since the first type of relations are hard wired via \texttt{TrueMethods}, there are not
meant to be switched on or off depending on the category. Contrary to this, the logic defined
by the relations from operations can be turned off or on for a specific \CapPkg category.
This can be achieved by using the commands \texttt{CapCategorySwitchLogicOn} and \texttt{CapCategorySwitchLogicOff}.
Generally it is not a good idea to use these switches mid-computation, since the \texttt{ToDoListEntries} for the
already computed objects remain and might still be executed. If one wants no logic for a specific category,
it is always best to switch off logic at creation time of the category.

\subsection{Adding own logic files}


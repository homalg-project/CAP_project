\CapPkg supports categorical constructions 
for a category $\mathbf{C}$
of the following types:

\begin{itemize}
 \item Basic constructions which only depend on the fact that $\mathbf{C}$ is category,
 such as composition ($\PreCompose$), identity ($\IdentityMorphism$), decisions whether
 a morphism is a mono, epi, or iso ($\IsMonomorphism, \IsEpimorphism, \IsIsomorphism$).
 \item Some limit and colimit constructions, such as direct product ($\DirectProduct$),
 coproduct ($\Coproduct$), fiber product ($\FiberProduct$), pushout ($\Pushout$),
 kernel ($\KernelObject$), cokernel ($\Cokernel$).
 \item Constructions for monoidal categories, such as tensor product ($\TensorProductOnObjects$, $\TensorProductOnMorphisms$),
 unitors ($\LeftUnitor$, $\RightUnitor$), associators \\($\AssociatorLeftToRight$, $\AssociatorRightToLeft$),
 internal hom ($\InternalHomOnObjects$, $\InternalHomOnMorphisms$).
\end{itemize}

\begin{notation}
 For the definition of a category, see chapter \ref{chapter:specifications}, section \ref{section:categories}.
\end{notation}


\section{Basic Constructions}

\section{Limits and Colimits}\label{section:universalobjects}

In \CapPkg, support is given for the implementation of some special kinds of limits and
colimits. We first give the definition of a limit and a colimit.
Let $\mathbf{C}$ be a category. Let $\mathbf{I}$ be another category (called an \textbf{index category}) and $D: \mathbf{I} \rightarrow \mathbf{C}$
a functor (called a \textbf{diagram}). 
\begin{definition} 
 A \textbf{source} of $D$ is a collection of morphisms $( s_i: S \rightarrow D_i)_{i \in \mathbf{I}}$ such that
 $\left( D( i \rightarrow j ) \circ s_i \right) \sim_{S,D_j} s_j$ for every arrow $(i \rightarrow j) \in \mathbf{I}$.
\end{definition}
 
\begin{definition}
 A triple $(L, \lambda, u)$ consisting of the following data:
 \begin{enumerate}
  \item an object $L \in \mathbf{C}$,
  \item a source $\lambda = ( \lambda_i: L \rightarrow D_i)_{i \in \mathbf{I}}$,
  \item a dependent function $u$ mapping every source $\tau = ( \tau_i: T \rightarrow D_i )_{i \in \mathbf{I}}$
        to a morphism $u( \tau ): T \rightarrow L$ such that
        $\lambda_i \circ u( \tau ) \sim_{T,D_i} \tau_i$,
 \end{enumerate}
 is called a \textbf{limit of the diagram $D$}, if the morphisms $u(\tau)$ are uniquely determined up to
 congruence of morphisms.
\end{definition}

\begin{definition}
 A \textbf{sink} of $D$ is a collection of morphisms $( s_i: D_i \rightarrow S )_{i \in \mathbf{I}}$
 such that $\left(s_i \circ D( i \rightarrow j )\right) \sim_{D_i, S} s_j$ for every arrow $(i \rightarrow j) \in \mathbf{I}$.
\end{definition}


\begin{definition}
 A triple $(C, c, u)$ consisting of the following data:
 \begin{enumerate}
  \item an object $C \in \mathbf{C}$,
  \item a sink $c = ( c_i: D_i \rightarrow C )_{i \in \mathbf{I}}$,
  \item a dependent function $u$ mapping every source $\tau = ( \tau_i: D_i \rightarrow T )_{i \in \mathbf{I}}$
        to a morphism $u( \tau ): C \rightarrow T$ such that
        $u(\tau) \circ c_{i} \sim_{D_i,T} \tau_i$,
 \end{enumerate}
 is called a \textbf{colimit of the diagram $D$}, if the morphisms $u(\tau)$ are uniquely determined up to
 congruence of morphisms.
\end{definition}

\section{Monoidal Categories}

\section{Derivations}\label{section:derivations}